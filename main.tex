\documentclass[twocolumn]{article}
\usepackage{graphicx}
\usepackage{algorithm,algorithmicx}
\usepackage{amsfonts,amssymb,amsmath,amsthm}

\newtheorem{theorem}{}[section]

\title{Multi-Target Shape Interpolation With Karcher Mean of Metrics}
\author{}
\date{}

\begin{document}

\maketitle

\begin{abstract}
    
\end{abstract}

Shape interpolation, Karcher mean, bounded distortion

\section{Introduction}

\section{Related Work}

\section{The Problem}

We first formalize the problem of multi-target planar shape interpolation. 
Consider a source planar triangular mesh $S_0 = (V_0, E, F)$ and a set of $n$ target meshes $S_i = (V_i, E, F)$ for $i = 1, \dots, n$, all sharing the same connectivity (edges $E$ and faces $F$) and face orientation. 
The embedding of each mesh is given by vertex positions $V_i \in \mathbb{R}^2$.

A natural way to represent the deformation from the source to each target is through a piecewise-linear map $\phi_i: S_0 \to S_i$ defined on vertices as $\phi_i(V_0) = V_i$ and extended linearly over faces via barycentric coordinates. 
On each face $f \in F$, the Jacobian matrix $J_{i,f} \in \mathbb{R}^{2 \times 2}$ of $\phi_i$ is constant and captures the local affine transformation.

The \emph{pullback metric} $g_{i,f}$ induced by $\phi_i$ on face $f$ is defined as
\begin{equation}
    g_{i,f} = J_{i,f}^\top J_{i,f},
\end{equation}
which is a symmetric positive-definite (SPD) $2\times 2$ matrix. 
This metric encodes how lengths and angles are distorted locally by the map. 
For the identity map (i.e., the source itself), the pullback metric is simply the identity matrix $I$.

From the eigenvalues $\lambda_{i,f,1} \geq \lambda_{i,f,2} > 0$ of $g_{i,f}$, we define two fundamental distortion measures:
\begin{align}
    K_{i,f} &= \sqrt{\frac{\lambda_{i,f,1}}{\lambda_{i,f,2}}}, \quad 
    D_{i,f} = \sqrt{\lambda_{i,f,1} \cdot \lambda_{i,f,2}}
\end{align}
The \emph{conformal distortion} $K_{i,f}$ measures angular deviation from a conformal map ($K=1$ indicates no angular distortion), while the \emph{area distortion} $D_{i,f}$ measures local scaling of area ($D=1$ indicates area preservation). 
Both quantities are defined per face.

Given a set of nonnegative weights $w = (w_0, w_1, \dots, w_n)$ with $\sum_{i=0}^n w_i = 1$, where $w_0$ corresponds to the source and $w_i$ to the $i$-th target, our goal is to produce an interpolated mesh $S_w$ that naturally blends the input shapes while controlling both conformal and area distortions. 
The core challenge is to define a meaningful blending of the pullback metrics $\{g_{i,f}\}$ that leads to a valid planar embedding with bounded distortion.

\section{Method}

Our goal is to interpolate between multiple planar shapes while maintaining bounded geometric distortion. 
To achieve this, we propose a metric blending approach that generalizes existing single-target interpolation schemes to multiple targets. 
We formulate the blending as a \emph{Karcher mean} problem on the Riemannian manifold of SPD matrices. 
This choice naturally extends the desirable properties of single-target interpolation to the multi-target case while providing strong theoretical guarantees on distortion bounds.

\subsection{Karcher Mean of Pullback Metrics}

Given a set of SPD matrices $\{g_i\}_{i=0}^n$ corresponding to the pullback metrics of $n+1$ shapes (including the source $g_0 = I$), and a set of weights $w = (w_0, \dots, w_n)$ with $w_i \geq 0$ and $\sum_i w_i = 1$, we define the blended metric $g_w$ as the solution to the Karcher mean equation:

\begin{equation}
    \sum_{i=0}^n w_i \, \log\left( g_w^{-1/2} g_i g_w^{-1/2} \right) = 0,
    \label{eq:karcher}
\end{equation}

This equation arises from minimizing the weighted sum of squared Riemannian distances:

\begin{equation}
    g_w = \arg\min_{g \succ 0} \sum_{i=0}^n w_i \, d_{\text{Riem}}^2(g, g_i),
\end{equation}

with $d_{\text{Riem}}(A, B) = \|\log(A^{-1/2} B A^{-1/2})\|_F$.

The solution $g_w$ to \eqref{eq:karcher} exists and is unique for nonnegative weights, ensuring a well-defined blending operation. We denote this solution compactly as:

\begin{equation}
    g_w = G(w_0, \dots, w_n; g_0, \dots, g_n).
\end{equation}

For the special case of a single target ($n=1$), the Karcher mean  reduces to the familiar logarithmic blending, and we write:

\[
G(1-t, t; I, g_1) = g_1^t.
\]

This shows that our multi-target blending framework generalizes the single-target logarithmic blending, preserving its desirable distortion properties while extending to arbitrary numbers of input shapes.

\subsection{Basic Properties of Karcher Mean Blending}

The Karcher mean blending inherits several desirable properties from its variational formulation:

\textbf{Lagrange property.} If $w$ is a vector with $w_i=1$, then $g_w = g_i$. 

\textbf{Smoothness.} The mapping $w \mapsto g_w$ is smooth for $w_i > 0$, as the Karcher mean varies smoothly with weights on the SPD manifold.

\textbf{Symmetry.} The Karcher mean is invariant under permutation of the inputs: if $\sigma$ is a permutation of $\{0,\dots,n\}$, then $G(w_0,\dots,w_n; g_0,\dots,g_n) = G(w_{\sigma(0)},\dots,w_{\sigma(n)}; g_{\sigma(0)},\dots,g_{\sigma(n)})$.

These properties ensure that our interpolation scheme behaves intuitively and robustly under weight changes and input reordering.

\subsection{Bounded Conformal Distortion}

Let $K_i$ denote the conformal distortion associated with metric $g_i$, and $K_w$ the distortion of the blended metric $g_w$. We now prove that the Karcher mean blending yields a conformal distortion that is bounded by the geometric mean of the input distortions.

\begin{theorem}[Conformal distortion bound]
For any weights $w_i \geq 0$ with $\sum_i w_i = 1$, the conformal distortion $K_w$ of the Karcher mean $g_w$ satisfies
\begin{equation}
    K_w \leq \prod_{i=0}^n K_i^{w_i}.
    \label{eq:conformal-bound}
\end{equation}
\end{theorem}

\begin{proof}
Let $\lambda_{\max}(g)$ and $\lambda_{\min}(g)$ denote the largest and smallest eigenvalues of an SPD matrix $g$. 

From the variational characterization of eigenvalues and the Karcher mean definition, we can derive:
\begin{align}
    \lambda_{\max}(g_w) &\leq \prod_{i=0}^n \lambda_{\max}(g_i)^{w_i}, \\
    \lambda_{\min}(g_w) &\geq \prod_{i=0}^n \lambda_{\min}(g_i)^{w_i}.
\end{align}
Dividing these inequalities and taking square roots yields \eqref{eq:conformal-bound}.
\end{proof}

\subsection{Bounded Area Distortion}

The area distortion $D$ is governed by the determinant of the metric, which exhibits a particularly elegant behavior under Karcher mean blending.

\begin{theorem}[Area distortion product formula]
The area distortion $D_w$ of the Karcher mean $g_w$ satisfies
\begin{equation}
    D_w = \prod_{i=0}^n D_i^{w_i}.
    \label{eq:area-equality}
\end{equation}
\end{theorem}

\begin{proof}
Taking the trace of both sides of the Karcher equation \eqref{eq:karcher} and using the identity $\operatorname{tr}(\log A) = \log\det A$ for SPD matrices, we obtain:
\begin{align*}
    0 &= \sum_{i=0}^n w_i \operatorname{tr}\left(\log\left(g_w^{-1/2} g_i g_w^{-1/2}\right)\right) \\
      &= \sum_{i=0}^n w_i \log\det\left(g_w^{-1/2} g_i g_w^{-1/2}\right) \\
      &= \sum_{i=0}^n w_i \left(\log\det g_i - \log\det g_w\right).
\end{align*}
Rearranging gives $\log\det g_w = \sum_i w_i \log\det g_i$, or equivalently,
\[
\det g_w = \prod_{i=0}^n (\det g_i)^{w_i}.
\]
Since $D = \sqrt{\det g}$, we obtain \eqref{eq:area-equality}.
\end{proof}

\section{Implementation}

\subsection{Blending}

\subsection{Realization}

\section{Results}

\section{Discussion and Conclusion}

\section{Acknowledgments}

\end{document}

